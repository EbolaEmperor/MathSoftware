\documentclass[UTF8]{ctexart}

\usepackage{graphicx}
\usepackage{amsmath}
\usepackage{amsfonts}
\usepackage{amsthm}
\usepackage{xeCJK}
\usepackage{graphicx} %插入图片的宏包
\usepackage{float} %设置图片浮动位置的宏包
\usepackage{subfigure} %插入多图时用子图显示的宏包
\usepackage{geometry}
\usepackage{enumitem}
\usepackage{verbatimbox}
\usepackage[ruled,linesnumbered]{algorithm2e}
\usepackage[colorlinks,linkcolor=blue,citecolor=blue]{hyperref}
\usepackage{cite}

\newtheorem{example}{例}             % 整体编号
\newtheorem{theorem}{定理}[section]  % 按 section 编号
\newtheorem{definition}{定义}
\newtheorem{axiom}{公理}
\newtheorem{property}{性质}
\newtheorem{proposition}{命题}
\newtheorem{lemma}{引理}
\newtheorem{corollary}{推论}
\newtheorem{remark}{注解}
\newtheorem{condition}{条件}
\newtheorem{conclusion}{结论}
\newtheorem{assumption}{假设}

\geometry{a4paper,scale=0.8}
\CTEXsetup[format={\Large\bfseries}]{section}

\title{\textbf{roots.c分析}}

\author{\CJKfamily{kai} 黄文\hbox{\scalebox{0.6}[1]{羽}\kern-.1em\scalebox{0.5}[1]{中}}\\3200100006}

\begin{document}

\maketitle

\section{功能概述}

此代码用Brent方法求函数$f(x)=x^2-5$的根, 从而得到$\sqrt{5}$的近似值. 

\section{技术细节}

第一部分代码如下:

\begin{verbatim}
    int status;
    int iter = 0, max_iter = 100;
    const gsl_root_fsolver_type *T;
    gsl_root_fsolver *s;
    double r = 0, r_expected = sqrt (5.0);
    double x_lo = 0.0, x_hi = 5.0;
    gsl_function F;
    struct quadratic_params params = {1.0, 0.0, -5.0};
  
    F.function = &quadratic;
    F.params = &params;

    T = gsl_root_fsolver_brent;
    s = gsl_root_fsolver_alloc (T);
    gsl_root_fsolver_set (s, &F, x_lo, x_hi);
\end{verbatim}

这一部分代码调用了 \verb | demo_fn.c | 中实现的二次函数, 将 \verb |F.function| 定义为一个二次函数$f(x)=(ax+b)x+c$, 其中$a=1,b=0,c=-5$, 即为$f(x)=x^2-5$. 

另外还初始化了一个Brent方法迭代器, 初始区间为$[0,5]$, 最大迭代次数限制为100. 

第二部分代码如下:

\begin{verbatim}
    do
    {
      iter++;
      status = gsl_root_fsolver_iterate (s);
      r = gsl_root_fsolver_root (s);
      x_lo = gsl_root_fsolver_x_lower (s);
      x_hi = gsl_root_fsolver_x_upper (s);
      status = gsl_root_test_interval (x_lo, x_hi,
                                       0, 0.001);

      if (status == GSL_SUCCESS)
        printf ("Converged:\n");

      printf ("%5d [%.7f, %.7f] %.7f %+.7f %.7f\n",
              iter, x_lo, x_hi,
              r, r - r_expected, 
              x_hi - x_lo);
    }
  while (status == GSL_CONTINUE && iter < max_iter);

  gsl_root_fsolver_free (s);
\end{verbatim}

这一部分代码是一个迭代过程. 每一次迭代分为计算当前步根、计算下一步区间范围、判断收敛、输出当前步信息四个步骤. 停机准则为超过最大迭代次数或达到收敛精度. 最后将迭代器的空间释放回收. 

\end{document}